% Options for packages loaded elsewhere
\PassOptionsToPackage{unicode}{hyperref}
\PassOptionsToPackage{hyphens}{url}
%
\documentclass[
]{article}
\usepackage{lmodern}
\usepackage{amssymb,amsmath}
\usepackage{ifxetex,ifluatex}
\ifnum 0\ifxetex 1\fi\ifluatex 1\fi=0 % if pdftex
  \usepackage[T1]{fontenc}
  \usepackage[utf8]{inputenc}
  \usepackage{textcomp} % provide euro and other symbols
\else % if luatex or xetex
  \usepackage{unicode-math}
  \defaultfontfeatures{Scale=MatchLowercase}
  \defaultfontfeatures[\rmfamily]{Ligatures=TeX,Scale=1}
\fi
% Use upquote if available, for straight quotes in verbatim environments
\IfFileExists{upquote.sty}{\usepackage{upquote}}{}
\IfFileExists{microtype.sty}{% use microtype if available
  \usepackage[]{microtype}
  \UseMicrotypeSet[protrusion]{basicmath} % disable protrusion for tt fonts
}{}
\makeatletter
\@ifundefined{KOMAClassName}{% if non-KOMA class
  \IfFileExists{parskip.sty}{%
    \usepackage{parskip}
  }{% else
    \setlength{\parindent}{0pt}
    \setlength{\parskip}{6pt plus 2pt minus 1pt}}
}{% if KOMA class
  \KOMAoptions{parskip=half}}
\makeatother
\usepackage{xcolor}
\IfFileExists{xurl.sty}{\usepackage{xurl}}{} % add URL line breaks if available
\IfFileExists{bookmark.sty}{\usepackage{bookmark}}{\usepackage{hyperref}}
\hypersetup{
  pdftitle={Ejercicios Tema 3 - Distribuciones Notables: continuas},
  pdfauthor={Ricardo Alberich, Juan Gabriel Gomila y Arnau Mir},
  hidelinks,
  pdfcreator={LaTeX via pandoc}}
\urlstyle{same} % disable monospaced font for URLs
\usepackage[margin=1in]{geometry}
\usepackage{graphicx,grffile}
\makeatletter
\def\maxwidth{\ifdim\Gin@nat@width>\linewidth\linewidth\else\Gin@nat@width\fi}
\def\maxheight{\ifdim\Gin@nat@height>\textheight\textheight\else\Gin@nat@height\fi}
\makeatother
% Scale images if necessary, so that they will not overflow the page
% margins by default, and it is still possible to overwrite the defaults
% using explicit options in \includegraphics[width, height, ...]{}
\setkeys{Gin}{width=\maxwidth,height=\maxheight,keepaspectratio}
% Set default figure placement to htbp
\makeatletter
\def\fps@figure{htbp}
\makeatother
\setlength{\emergencystretch}{3em} % prevent overfull lines
\providecommand{\tightlist}{%
  \setlength{\itemsep}{0pt}\setlength{\parskip}{0pt}}
\setcounter{secnumdepth}{-\maxdimen} % remove section numbering

\title{Ejercicios Tema 3 - Distribuciones Notables: continuas}
\author{Ricardo Alberich, Juan Gabriel Gomila y Arnau Mir}
\date{Curso de Probabilidad y Variables Aleatorias con R y Python}

\begin{document}
\maketitle

\hypertarget{ejercicios-distribuciones-notables-continuas}{%
\section{Ejercicios distribuciones notables
continuas}\label{ejercicios-distribuciones-notables-continuas}}

\begin{enumerate}
\def\labelenumi{\arabic{enumi}.}
\item
  Si \(X\) está distribuida uniformemente en \((0,2)\) e \(Y\) es una
  variable exponencial con parámetro \(\lambda\). Calcular el valor de
  \(\lambda\) tal que \(P(X<1)=P(Y<1)\).
\item
  El tiempo \(X\) que necesita un corredor de fondo para recorrer un
  kilómetro es una variable normal con parámetros \(\mu =3\) minutos 45
  segundos y \(\sigma = 10\) segundos.
\end{enumerate}

\begin{itemize}
\item
  \begin{enumerate}
  \def\labelenumi{\alph{enumi})}
  \tightlist
  \item
    ¿Cuál es la probabilidad de que este atleta recorra la milla en
    menos de 4 minutos?\\
  \end{enumerate}
\item
  \begin{enumerate}
  \def\labelenumi{\alph{enumi})}
  \setcounter{enumi}{1}
  \tightlist
  \item
    ¿Y en más de 3.5 minutos?
  \end{enumerate}
\end{itemize}

\begin{enumerate}
\def\labelenumi{\arabic{enumi}.}
\setcounter{enumi}{2}
\tightlist
\item
  Las marcas de un atleta de salto de altura siguen, aproximadamente,
  una ley normal con \(\mu=2\) Y \(\sigma=0.8\)
\end{enumerate}

\begin{itemize}
\item
  \begin{enumerate}
  \def\labelenumi{\alph{enumi})}
  \tightlist
  \item
    ¿Cuál es la altura máxima que puede saltar con una probabilidad del
    \(0.95\)
  \end{enumerate}
\item
  \begin{enumerate}
  \def\labelenumi{\alph{enumi})}
  \setcounter{enumi}{1}
  \tightlist
  \item
    ¿Cuál es la altura que salta en el 40\% de los peores intentos?
  \end{enumerate}
\end{itemize}

\begin{enumerate}
\def\labelenumi{\arabic{enumi}.}
\setcounter{enumi}{3}
\tightlist
\item
  Un centro de atención telefónica por voz (\emph{call center}) recibe
  por termino medio 100 llamadas por hora. Suponed que el número de
  llamadas sigue un proceso Poisson
\end{enumerate}

\begin{itemize}
\item
  \begin{enumerate}
  \def\labelenumi{\alph{enumi})}
  \tightlist
  \item
    Sea X el tiempo entre dos llamadas consecutivas ¿cuál es la
    distribución de X?
  \end{enumerate}
\item
  \begin{enumerate}
  \def\labelenumi{\alph{enumi})}
  \setcounter{enumi}{1}
  \tightlist
  \item
    Calcular la probabilidad que pasen al menos 25 minutos hasta recibir
    la primera llamada.
  \end{enumerate}
\item
  \begin{enumerate}
  \def\labelenumi{\alph{enumi})}
  \setcounter{enumi}{2}
  \tightlist
  \item
    Calcular la probabilidad que pasen al menos 30 minutos hasta recibir
    la segunda llamada.
  \end{enumerate}
\item
  \begin{enumerate}
  \def\labelenumi{\alph{enumi})}
  \setcounter{enumi}{3}
  \tightlist
  \item
    Si denotamos por \(p_k\) la probabilidad de que pasen al menos 30
    minutos hasta recibir la \(k-\)ésima llamada, encontrar la relación
    que hay entre \(p_k\) y \(p_{k-1}\).
  \end{enumerate}
\end{itemize}

\begin{enumerate}
\def\labelenumi{\arabic{enumi}.}
\setcounter{enumi}{4}
\item
  Sea \(X\) una variable aleatoria normal con parámetros \(\mu=1\) y
  \(\sigma=1\). Calculad el valor de \(b\) tal que
  \(P\left((X-1)^2\leq b\right)=0.1\).
\item
  Sea \(Z\) una variable aleatoria \(N(0,1)\). Calcular
  \(P\left(\left(Z-\frac{1}{4}\right)^2 >\frac{1}{16}\right)\).
\item
  Las peticiones a un servidor informático llegan a un ritmo de medio de
  15 peticiones por segundo. Sabemos que el nombre de peticiones que
  llegan en un segundo es una variable aleatoria de Poisson.
\end{enumerate}

\begin{itemize}
\item
  \begin{enumerate}
  \def\labelenumi{\alph{enumi})}
  \tightlist
  \item
    Calcular la probabilidad que no lleguen peticiones en un segundo.
  \end{enumerate}
\item
  \begin{enumerate}
  \def\labelenumi{\alph{enumi})}
  \setcounter{enumi}{1}
  \tightlist
  \item
    Calcular la probabilidad que lleguen más de 10 peticiones en un
    segundo.
  \end{enumerate}
\end{itemize}

\begin{enumerate}
\def\labelenumi{\arabic{enumi}.}
\setcounter{enumi}{7}
\tightlist
\item
  Tenemos que elegir entre dos tarjetas gráficas (TG1 y TG2) para
  entrenar su red neuronal. El tiempo de vida del la TG1 se ha modelado
  según una \(N(\mu_1=120000, \sigma_1=140000)\) (la probabilidad de un
  tiempo de vida negativo es despreciable) y en TG2 según una
  \(N(\mu_2=22000, \sigma_2=1000)\).
\end{enumerate}

\begin{itemize}
\item
  \begin{enumerate}
  \def\labelenumi{\alph{enumi})}
  \tightlist
  \item
    ¿Qué tarjeta elegimos si el tiempo de duración objetivo del sistemas
    es de 20000 horas?
  \end{enumerate}
\item
  \begin{enumerate}
  \def\labelenumi{\alph{enumi})}
  \setcounter{enumi}{1}
  \tightlist
  \item
    ¿Y si es de 24000 horas?
  \end{enumerate}
\end{itemize}

\end{document}
